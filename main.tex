\documentclass{beamer}
\usepackage{listings}
\usepackage{xcolor}
\usepackage{hyperref}

\title{Server Administration \newline with NixOS}
\author{by mrtuxa}
\date{2023}

\usepackage{listings}
\usepackage{xcolor}

\usepackage{booktabs}
\usepackage{tabularx}
\usepackage{lipsum}

\definecolor{codegreen}{rgb}{0,0.6,0}
\definecolor{codegray}{rgb}{0.5,0.5,0.5}
\definecolor{codepurple}{rgb}{0.58,0,0.82}
\definecolor{backcolour}{rgb}{0.95,0.95,0.92}

\lstdefinestyle{mystyle}{
    backgroundcolor=\color{backcolour},   
    commentstyle=\color{codegreen},
    keywordstyle=\color{magenta},
    numberstyle=\tiny\color{codegray},
    stringstyle=\color{codepurple},
    basicstyle=\ttfamily\footnotesize,
    breakatwhitespace=false,         
    breaklines=true,                 
    captionpos=b,                    
    keepspaces=true,                 
    numbers=left,                    
    numbersep=5pt,                  
    showspaces=false,                
    showstringspaces=false,
    showtabs=false,                  
    tabsize=2
}

\lstset{style=mystyle}

\begin{document}

\frame{\titlepage}

\begin{frame}
\frametitle{Contact}

\href{http://www.latex-tutorial.com}{Matrix: @mrtuxa:sys2nix.de} \\
\href{https://zug.network/@mrtuxa}{Mastodon: @mrtuxa@zug.network} \\
\href{mailto:mrtuxa@infra-sys.de}{E-Mail: mrtuxa@infra-sys.de} \\
\href{mailto:mrtuxa@leipzig.freifunk.net}{E-Mail: mrtuxa@leipzig.freifunk.net}
    
\end{frame}

\begin{frame}[fragile]
\frametitle{Nix Flakes}

\begin{lstlisting}[language=c++]
{
  {
  	inputs = {
    		nixpkgs.url = "github:nixos/nixpkgs/nixos-23.05";
    		unstable.url = "github:nixos/nixpkgs/nixos-unstable";
    		home-manager = {
      		url = "github:nix-community/home-manager";
      		inputs.nixpkgs.follows = "nixpkgs";
    		};
  	};
  }
}
\end{lstlisting}

\end{frame}


\begin{frame}[fragile]
\frametitle{Nix Flakes}

\begin{lstlisting}[language=c++]
{
  outputs = inputs @ {
    nixpkgs,
    unstable,
    home-manager,
    ...
  }: {
    nixosConfigurations = {
      mrtuxaServer = unstable.lib.nixosSystem {
				system = "x86_64-linux";
        modules = [./hosts/dezentrale];
      };
    };
  };
}
\end{lstlisting}

\end{frame}

\begin{frame}[fragile]
\frametitle{Nix Expressions File Structure}

\begin{lstlisting}[language=bash]

flake.lock
flake.nix
hosts/
	dezentrale/
		default.nix
		hardware-configuration.nix
modules/
	reverse-proxy.nix
secrets/
	homeserver.yaml
	vapidPrivateKey
services/
	communication/
		default.nix
		mail.nix
		turn.nix
		social/

\end{lstlisting}

\end{frame}



\begin{frame}[fragile]
\frametitle{Nix Expressions File Structure}

\begin{lstlisting}[language=bash]
		social/
			mastodon.nix
			matrix.nix
website/
	# static pages

\end{lstlisting}

\end{frame}


\begin{frame}[fragile]
\frametitle{Nix Expressions File Structure}
\begin{table}
\begin{tabularx}{\linewidth}{>{\parskip1ex}X@{\kern4\tabcolsep}>{\parskip1ex}X}
\toprule
\hfil\bfseries Pros
&
\hfil\bfseries Cons
\\\cmidrule(r{3\tabcolsep}){1-1}\cmidrule(l{-\tabcolsep}){2-2}

%% PROS, seperated by empty line or \par
Nixos\par
ist\par
cool
&

%% CONS, seperated by empty line or \par
Nixos\par
ist\par
uncool

\\\bottomrule
\end{tabularx}
\end{table}
\end{frame}

\end{document}
